\documentclass[]{article}
\usepackage{lmodern}
\usepackage{amssymb,amsmath}
\usepackage{ifxetex,ifluatex}
\usepackage{fixltx2e} % provides \textsubscript
\ifnum 0\ifxetex 1\fi\ifluatex 1\fi=0 % if pdftex
  \usepackage[T1]{fontenc}
  \usepackage[utf8]{inputenc}
\else % if luatex or xelatex
  \ifxetex
    \usepackage{mathspec}
  \else
    \usepackage{fontspec}
  \fi
  \defaultfontfeatures{Ligatures=TeX,Scale=MatchLowercase}
\fi
% use upquote if available, for straight quotes in verbatim environments
\IfFileExists{upquote.sty}{\usepackage{upquote}}{}
% use microtype if available
\IfFileExists{microtype.sty}{%
\usepackage{microtype}
\UseMicrotypeSet[protrusion]{basicmath} % disable protrusion for tt fonts
}{}
\usepackage[margin=1in]{geometry}
\usepackage{hyperref}
\hypersetup{unicode=true,
            pdftitle={Methodological Supplement Study 2},
            pdfauthor={Nicolas Restrepo},
            pdfborder={0 0 0},
            breaklinks=true}
\urlstyle{same}  % don't use monospace font for urls
\usepackage{graphicx}
% grffile has become a legacy package: https://ctan.org/pkg/grffile
\IfFileExists{grffile.sty}{%
\usepackage{grffile}
}{}
\makeatletter
\def\maxwidth{\ifdim\Gin@nat@width>\linewidth\linewidth\else\Gin@nat@width\fi}
\def\maxheight{\ifdim\Gin@nat@height>\textheight\textheight\else\Gin@nat@height\fi}
\makeatother
% Scale images if necessary, so that they will not overflow the page
% margins by default, and it is still possible to overwrite the defaults
% using explicit options in \includegraphics[width, height, ...]{}
\setkeys{Gin}{width=\maxwidth,height=\maxheight,keepaspectratio}
\IfFileExists{parskip.sty}{%
\usepackage{parskip}
}{% else
\setlength{\parindent}{0pt}
\setlength{\parskip}{6pt plus 2pt minus 1pt}
}
\setlength{\emergencystretch}{3em}  % prevent overfull lines
\providecommand{\tightlist}{%
  \setlength{\itemsep}{0pt}\setlength{\parskip}{0pt}}
\setcounter{secnumdepth}{0}
% Redefines (sub)paragraphs to behave more like sections
\ifx\paragraph\undefined\else
\let\oldparagraph\paragraph
\renewcommand{\paragraph}[1]{\oldparagraph{#1}\mbox{}}
\fi
\ifx\subparagraph\undefined\else
\let\oldsubparagraph\subparagraph
\renewcommand{\subparagraph}[1]{\oldsubparagraph{#1}\mbox{}}
\fi

%%% Use protect on footnotes to avoid problems with footnotes in titles
\let\rmarkdownfootnote\footnote%
\def\footnote{\protect\rmarkdownfootnote}

%%% Change title format to be more compact
\usepackage{titling}

% Create subtitle command for use in maketitle
\providecommand{\subtitle}[1]{
  \posttitle{
    \begin{center}\large#1\end{center}
    }
}

\setlength{\droptitle}{-2em}

  \title{Methodological Supplement Study 2}
    \pretitle{\vspace{\droptitle}\centering\huge}
  \posttitle{\par}
    \author{Nicolas Restrepo}
    \preauthor{\centering\large\emph}
  \postauthor{\par}
      \predate{\centering\large\emph}
  \postdate{\par}
    \date{4/2/2020}

\usepackage{booktabs}
\usepackage{longtable}
\usepackage{array}
\usepackage{multirow}
\usepackage{wrapfig}
\usepackage{float}
\usepackage{colortbl}
\usepackage{pdflscape}
\usepackage{tabu}
\usepackage{threeparttable}
\usepackage{threeparttablex}
\usepackage[normalem]{ulem}
\usepackage{makecell}
\usepackage{xcolor}

\begin{document}
\maketitle

\hypertarget{model-selection}{%
\subsection{Model Selection}\label{model-selection}}

Here, I am going to go through the specifications of the models I fitted
and to show the reasoning behind my model selection. To answer this
question, I fitted three different models. In all of them, the outcome
variable is the time it took participants to classify an event as
harmful or harmless. The main independent variable is Euclidean distance
from the prototypical moral wrong and I control for the length of the
statement and its readability (using the Felsch Kincaid index). All
models are multi-level, cross-classified models, including varying
intercepts for respondents and scenarios. The main difference is how the
term for distance from the prototype is expressed: the first model
includes only a linear term; the second expresses the relationship in a
quadratic manner; and the third describes the relationship
logarithmically.

Here, I will print the results of all three models:

Next, I compare the WAIC values of each of the specifications:

The comparison shows that the quadratic model is preferable, but only by
a slight margin. The implication is that information criteria cannot be
our only tool for discerning which model is more appropriate. We need to
rely on more theoretically informed criteria. This is why I base my
decision partly on the relationship between reaction time and proportion
of harmfulness that is shown in the main body of the paper.

\hypertarget{model-excluding-purity-scenarios}{%
\subsection{Model excluding purity
scenarios}\label{model-excluding-purity-scenarios}}

One of the concerns of the analaysis above is that the quadratic
relationship is mainly driven by the so-called purity violations. Here,
I fit the quadratic model without these datapoints to show that the
relationship still holds. The coefficients of the model are the
following:

~

reaction time harm

Predictors

Estimates

CI

p

(Intercept)

0.18

0.05~--~0.31

0.006

distances\_prot

0.16

0.07~--~0.26

0.001

distances\_prot\^{}2

-0.13

-0.22~--~-0.03

0.008

length

0.17

0.04~--~0.29

0.008

readibility\_indices

-0.08

-0.19~--~0.04

0.195

Random Effects

σ2

0.71

τ00 ids

0.25

τ00 scenario

0.03

ICC

0.28

N ids

184

N scenario

20

Observations

3680

Marginal R2 / Conditional R2

0.053 / 0.318


\end{document}
